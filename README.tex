\documentclass[12pt,a4paper]{article}
\usepackage[latin2]{inputenc}
\usepackage{graphicx}
\usepackage{ulem}
\usepackage{amsmath}
\usepackage[nottoc,numbib]{tocbibind}
\begin{document}

{\centering
\includegraphics[scale=0.5]{a.png}

\

\

{\huge \textbf{wizeUp}}


\

\textbf{Damir Bar}

\textbf{Eran Sterman}

\textbf{Shay Cohen}

\textbf{Sefi Shalom}\\

\

{\underline{Supervisor}}: Dr. Amit Dvir

\

August 2018

}


    \newpage

    \tableofcontents

    \newpage

    \section{Introduction}

    \subsection{Abstract}

    The educational system, up until the 20th century, hasn't evolved much. By the 00' the system improved massively by using a wide range of different services and applications to ease the learning experience, such as:

    \begin{itemize}
        \item WhatsApp, Facebook - Forum/messaging tools for group chatting,
        messy file sharing system and QA.
        \item Google Drive, Dropbox - Generic not student-oriented cloud system.
        Old communication system between students and lecturers, an exercise
        submission feature and QA option.
        \item Gool - Non-free on-demand video courses.
        \item Kahoot - Interactive student teacher app.
        \item Institution based feedback.
    \end{itemize}
    wizeUp is a reality which combines
    and enhances these features and more in to one easy-to-use user-friendly
    platform.



    \subsection{Related Work}


    This chapter surveys previous student-teacher platforms. Our project
    isn't about reinventing the wheel, but to suggest a decent way to bind
    between existing products and to innovate with our live streaming.
    We'll demonstrate for each field how it exists, but suffers lack of
    functionality and accessibility for students/teachers.

    \newcounter{numberedCntBD}
    \begin{enumerate}
        \item \textbf{\underline{Clouds}}
        \setcounter{numberedCntBD}{\theenumi}
    \end{enumerate}
    As we all know, clouds exist for a very long time. There are clouds
    owned by the strongest companies in the world. Our project isn't meant
    to compete with any of those, but to optimize the student/teacher
    approach to a study-relevant cloud. Our application will identify the
    file's extension, decide whether it's a document, a summary or a
    previous test, and will sort it so the user will enjoy surfing our
    cloud, instead of getting lost in it.



    \begin{enumerate}
        \setcounter{enumi}{\thenumberedCntBD}
        \item \textbf{\underline{Text-Message systems}}
        \setcounter{numberedCntBD}{\theenumi}
    \end{enumerate}
    There are lots of text-messaging systems, such as WhatsApp, Telegram,
    etc. Our platform will suggest a basic messaging system.





    \begin{enumerate}
        \setcounter{enumi}{\thenumberedCntBD}
        \item \textbf{\underline{GPA Calculator}}
        \setcounter{numberedCntBD}{\theenumi}
    \end{enumerate}
    During our studies, we came across several GPA calculators. One of them
    was in our university's mobile application and the other was an external
    mobile application. The problem with our university's application was
    the fact that it was unmodifiable. We couldn't calculate our "\,what
    ifs" or to check whether we should retake a certain course. The problem
    with the external application was the fact that it's external. It forced
    us to download a third-party application to our mobile phones.



    \begin{enumerate}
        \setcounter{enumi}{\thenumberedCntBD}
        \item \textbf{\underline{Tracking course progress}}
        \setcounter{numberedCntBD}{\theenumi}
    \end{enumerate}
    Our product provides students with a course progress tracking feature,
    based on taking quizzes, submitting assignments and reading papers. Our
    team has yet to encounter such a feature.



    \begin{enumerate}
        \setcounter{enumi}{\thenumberedCntBD}
        \item \textbf{\underline{Learning management system}}
        \setcounter{numberedCntBD}{\theenumi}
    \end{enumerate}
    There are many learning management systems. The one which the project's
    team used is Moodle \cite{moodle,usingmoodle}. Moodle suggest a lot, but as you will witness, we suggest more.

    \begin{enumerate}
        \setcounter{enumi}{\thenumberedCntBD}
        \item \textbf{\underline{Live Stream Video}}
        \setcounter{numberedCntBD}{\theenumi}
    \end{enumerate}
    This is where we truly innovate. Using the product of our project, you
    will be able to be at the lecture, even if you are absent from the
    class. You can send/receive messages from other students from the same
    lecture group during the lecture  \cite{onlinecourses}. You can ask the group questions and
    answer questions yourself.



    \begin{enumerate}
        \setcounter{enumi}{\thenumberedCntBD}
        \item \textbf{\underline{Live Feedback}}
        \setcounter{numberedCntBD}{\theenumi}
    \end{enumerate}
    Everyone benefits from the Live Feedback. Students which haven't quite
    gotten the last topic learned in class, will no longer desperately raise
    their hands, but post it right into our Live Feedback system in real
    time \cite{childrencomputer}. If most of the students felt the same way, the teacher might want
    to consider it. If a student did understand, he can tell the application
    that he did understand and by so to balance the "\,wizeScale".



    \begin{enumerate}
        \setcounter{enumi}{\thenumberedCntBD}
        \item \textbf{\underline{Statistical Gathering}}
        \setcounter{numberedCntBD}{\theenumi}
    \end{enumerate}
    The product of this project will independently gather statistical
    information regarding students' satisfaction and students' progress  \subsection{Functionalities}.
    This way the educational institution will be able to monitor its
    employees and improve.

    \newpage

    \subsection{Scope}\label{section:_Toc146371246}
    \begin{itemize}
        \item Create different users with varied roles and scopes.
        \item User Experience:
        \begin{itemize}
            \item Live stream video
            \item Tracking course progress
            \item Social network
            \item wizeUp's annual extraordinary
        \end{itemize}
        \item User convenience
        \begin{itemize}
            \item Grades Average calculator
            \item File system
            \item Q\&A
            \item Student answer providing in addition to teacher
            \item Android/IOS/Web UI
            \item Messaging system
            \item Statistical collection
            \begin{itemize}
                \item Live feedback
                \item Fame system
                \item Scoring system
            \end{itemize}
            \item Security
            \begin{itemize}
                \item Spamming/Abusive user reporting system
                \item Anonymity option
                \item General security
            \end{itemize}
            \item Accessibility
            \begin{itemize}
                \item Multiple language support
            \end{itemize}
        \end{itemize}
    \end{itemize}
    \newpage


    \subsection{Definitions, Acronyms and
    Abbreviation}\label{section:_Toc146371247}

    \begin{flushleft}

        \textbf{HTML} - Hyper Text Markup Language.

        \textbf{HTTP} - Hypertext Transfer Protocol.

        \textbf{Doc }- document.

        \textbf{MongoDB} - Document-oriented database program.

        \textbf{JavaScript }- Programming language

        \textbf{Node}.js - JavaScript run-time environment for executing
        JavaScript code server-side.

        \textbf{IDE }- Integrated Development Environment. A software
        application which provides comprehensive facilities to computer
        programmers for software development.

        \textbf{Android Studio} - The official IDE for Google's Android
        operating system.

        \textbf{WebStorm} - An IDE primarily for web development.

        \textbf{AngularJS} - JavaScript front-end web application framework.

        \textbf{JWT} - JSON Web Token. A tool for creating access tokens that assert some number of claims.

    \end{flushleft}


    \newpage


    \subsection{Technologies to be used}\label{section:_Toc146371249}


    \begin{itemize}

        \item Server API - Node.js
        \item Database Application - MongoDB
        \item Development tool - Android Studio and WebStorm
        \item Front-End framework - AngularJS
        \item Web Sockets - Socket.io
    \end{itemize}
    \newpage


    \subsection{Overview}
    wizeUp is a tool to help students get better at their studies and to
    help teachers have a better perspective on their class. wizeUp
    allows teachers to upload relevant files, answer students' questions and
    improve their teaching methods using live feedback. With
    wizeUp students can communicate with their classmates, share files
    and enjoy live video streaming, providing the option of live feedback.
    \newline
    wizeUp is the best way to gather statistical information for the use
    of both parties, students and teachers, as well as any department
    chosen. wizeUp will drive both sides, students and teachers, to
    thrive using a scoring system. wizeUp will make teaching and
    learning easier, faster and better.

    \newpage

    \section{Overall Description}

    \subsection{Product Perspective}

    \begin{figure}[h!]
        \includegraphics[width=\linewidth]{pp.jpg}
        \caption{high level architecture}
        \label{fig: figure}
    \end{figure}

    \newpage


    \subsection{Data Dictionary:}


    \newcounter{numberedCntBA}
    \begin{enumerate}
        \item \underline{User}
        \setcounter{numberedCntBA}{\theenumi}
    \end{enumerate}


    \begin{table}[h]
        \centering
        \begin{tabular}{|l|l|l|}
            \hline
            \textbf{Name} & \textbf{Type} & \textbf{Unique} \\
            \hline
            ID & Long & True \\
            \hline
            First name & String & False \\
            \hline
            Last name & String & False \\
            \hline
            Display Name & String & False \\
            \hline
            Email & String & True \\
            \hline
            About me & String & False \\
            \hline
            Country & String & False \\
            \hline
            City & String & False \\
            \hline
            Birthdate & Date & False \\
            \hline
            Gender & String & False \\
            \hline
            Courses & Object Array - Course & False \\
            \hline
            Photos & Array & False \\
            \hline
            File System & Object Array - references & True \\
            \hline
            Notifications & Object Array - Notification & True \\
            \hline
            Events & Object Array & True \\
            \hline
            Role & String & False \\
            \hline
            Register date & Date & False \\
            \hline
            Last updated & Date & False \\
            \hline
        \end{tabular}
    \end{table}


    \newpage




    \begin{enumerate}
        \setcounter{enumi}{\thenumberedCntBA}
        \item \underline{File:}
        \setcounter{numberedCntBA}{\theenumi}
    \end{enumerate}


    \begin{table}[h]
        \centering
        \begin{tabular}{|l|l|l|}
            \hline
            \textbf{Name} & \textbf{Type} & \textbf{Unique} \\
            \hline
            ID & Long & True \\
            \hline
            Name & String & False \\
            \hline
            UploaderID & String & False \\
            \hline
            URL & String & True \\
            \hline
            Creation Date & Date & False \\
            \hline
            Size & String & False \\
            \hline
            Type & String & False \\
            \hline
        \end{tabular}
    \end{table}


    \begin{enumerate}
        \setcounter{enumi}{\thenumberedCntBA}
        \item \underline{Notification:}
        \setcounter{numberedCntBA}{\theenumi}
    \end{enumerate}
    \begin{table}[h]
        \centering
        \begin{tabular}{|l|l|l|}
            \hline
            \textbf{Name} & \textbf{Type} & \textbf{Unique} \\
            \hline
            ReceiverID & String & False \\
            \hline
            SenderID & String & False \\
            \hline
            Creation Date & Date & False \\
            \hline
            Subject & String & False \\
            \hline
            Content & String & False \\
            \hline
            Is New & Boolean & False \\
            \hline
        \end{tabular}
    \end{table}


    \begin{enumerate}
        \setcounter{enumi}{\thenumberedCntBA}
        \item \underline{Course:}
        \setcounter{numberedCntBA}{\theenumi}
    \end{enumerate}


    \begin{table}[h]
        \centering
        \begin{tabular}{|l|l|l|}
            \hline
            \textbf{Name} & \textbf{Type} & \textbf{Unique} \\
            \hline
            Course number & Long & True \\
            \hline
            Name & String & False \\
            \hline
            Teacher & Teacher & False \\
            \hline
            Students & Object Array - String & False \\
            \hline
            File System & Object Array - references & True \\
            \hline
            Location & String & False \\
            \hline
            Sessions & Object Array - String & True \\
            \hline
            Quiz & Object Array & True \\
            \hline
            Test & Test & True \\
            \hline
            Exercises & Object Array & True \\
            \hline
            Points & Float & False \\
            \hline
            Messages & Object Array - Message & True \\
            \hline
            Register date & Date & False \\
            \hline
            Last updated & Date & False \\
            \hline
        \end{tabular}
    \end{table}




    \begin{enumerate}
        \setcounter{enumi}{\thenumberedCntBA}

        \newpage

        \item \underline{Test:}
        \setcounter{numberedCntBA}{\theenumi}
    \end{enumerate}
    \begin{table}[h]
        \centering
        \begin{tabular}{|l|l|l|}
            \hline
            \textbf{Name} & \textbf{Type} & \textbf{Unique} \\
            \hline
            ID & Long & True \\
            \hline
            Name & String & False \\
            \hline
            Teacher & Teacher & False \\
            \hline
            Students & Object Array - Student & False \\
            \hline
            Location & String & False \\
            \hline
            Dates & Date Array (2 elements) & False \\
            \hline
        \end{tabular}
    \end{table}


    \newpage

    \begin{enumerate}
        \setcounter{enumi}{\thenumberedCntBA}
        \item \underline{Session:}
        \setcounter{numberedCntBA}{\theenumi}
    \end{enumerate}

    \begin{table}[h]
        \centering
        \begin{tabular}{|l|l|l|}
            \hline
            \textbf{Name} & \textbf{Type} & \textbf{Unique} \\
            \hline
            ID & Long & True \\
            \hline
            Name & String & False \\
            \hline
            Teacher & Teacher & False \\
            \hline
            Students & Object Array - String & False \\
            \hline
            Location & String & False \\
            \hline
            Creation Date & Date & False \\
            \hline
            Messages & Object Array - String & False \\
            \hline
            Likers & Object Array - String & False \\
            \hline
            Dislikers & Object Array - String & False \\
            \hline
            Video URL & String & True \\
            \hline
            Course & String & False \\
            \hline
        \end{tabular}
    \end{table}


    \newpage




    \section{Functional Requirements}

    \begin{flushleft}

        \textbf{3.1}\ \ \textbf{User Class 1 - The Student}

        \

        \textbf{3.1.1 Functional requirement 1.1}

        \textbf{ID: FR1}

        TITLE: Download the mobile application

        DESC: A student should be able to download the mobile application
        through either an application store or similar service on the mobile
        phone. The application should be free of charge.
        RAT: For a student to download the mobile application.
        DEP: None

        \


        \textbf{3.1.2 Functional requirement 1.2}

        \textbf{ID: FR2}

        TITLE: Download and notify students of new releases

        DESC: When a new/updated version or release of the software is released,
        the student should check for these manually. The download of the new
        release should be done through the mobile phone in the same way as
        downloading the mobile application.

        RAT: For a student to download the most updated release.

        DEP: FR1

        \

        \textbf{3.1.3 Functional requirement 1.3}

        \textbf{ID: FR3}

        TITLE: Student registration - Mobile and Web application

        DESC: Given that a student has downloaded the mobile application or has
        browsed to the platform's homepage, then the student should be able to
        register through the mobile application or the website. The student must
        provide first name, last name, password, e-mail address and university.

        RAT: For a student to register on the mobile application or the website.

        DEP: FR1 for Mobile application, None for Web application.


        \
\newpage


        \textbf{3.1.4 Functional requirement 1.4}

        \textbf{ID: FR4}

        TITLE: Student log-in - Mobile and web application

        DESC: Given that a student has registered, then the student should be
        able to log into the application. The student's log-in information will
        be stored on the phone or browser, and in the future the student should
        be logged in automatically.

        RAT: For a student to be logged-in after registration.

        DEP: FR1, FR3


        \

        \textbf{3.1.5 Functional requirement 1.5}

        \textbf{ID: FR5}

        TITLE: Retrieve password

        DESC: Given that a student has registered, then the student should be
        able to retrieve his/her password by e-mail.

        RAT: For a student to retrieve his/her password.

        DEP: FR4

        \

        \textbf{3.1.6 Functional requirement 1.6}

        \textbf{ID: FR6}

        TITLE: Mobile/Web application - Homepage

        DESC: Given that a student is logged in to the application, then the
        first page that is shown should be the homepage, with the student's
        profile picture on the navigation bar.

        RAT: For a student to see his/her personal homepage.

        DEP: FR4

        \

        \textbf{3.1.7 Functional requirement 1.7}

        \textbf{ID: FR7}

        TITLE: Personal profile page

        DESC: Given that a student is logged in, then the student should be able
        to view his/her personal profile page.

        RAT: For a student to see his/her personal profile page.

        DEP: FR4


        \
\newpage


        \textbf{3.1.8 Functional requirement 1.8}

        \textbf{ID: FR8}

        TITLE: Personal course page

        DESC: Given that a student is logged in, then the student should be able
        to view his/her personal course page.

        RAT: For a student to see his/her personal course page.

        DEP: FR4

        \

        \textbf{3.1.9 Functional requirement 1.9}

        \textbf{ID: FR9}

        TITLE: Create Session

        DESC: Given that a student is logged in, the student should be able to create a session for other students to connect to.

        RAT: For a student to create sessions.

        DEP: FR4

        \

        \textbf{3.1.10 Functional requirement 1.10}

        \textbf{ID: FR10}

        TITLE: Connect to Session

        DESC: Given that a student is logged in, the student should be able to connect to a session.

        RAT: For a student to connect to sessions.

        DEP: FR4, FR9

        \

        \textbf{3.1.11 Functional requirement 1.11}

        \textbf{ID: FR11}

        TITLE: Free-text search

        DESC: Given that a student is logged in, then the student should be able
        to conduct a search by providing either course name, first name, last
        name or university name. The results are displayed in a list view by
        default.

        RAT: For a student to search through the free-text search.

        DEP: FR4

        \

        \textbf{3.1.12 Functional requirement 1.12}

        \textbf{ID: FR12}

        TITLE: Sorting results

        DESC: Given that a student has conducted a search, the student should be
        able to sort the results according to the result type: teacher, student,
        course or university.

        RAT: For a student to sort his/her search.

        DEP: FR11

        \

        \textbf{3.1.13 Functional requirement 1.13}

        \textbf{ID: FR13}

        TITLE: Filtering results

        DESC: Given that a student has conducted a search, the student should be
        able to filter the results according to the result type: teacher,
        student, course or university.

        RAT: For a student to filter his/her search.

        DEP: FR11

        \

        \textbf{3.1.14 Functional requirement 1.14}

        \textbf{ID: FR14}

        TITLE: Personal teachers list page

        DESC: Given that a student is logged in, then the student should be able
        to view his/her personal teachers list page, sorted by department and
        course.

        RAT: For a student to see his/her personal teachers list page.

        DEP: FR4

        \

        \textbf{3.1.15 Functional requirement 1.15}

        \textbf{ID: FR15}

        TITLE: File Management

        DESC: Given that a student is logged in, the student should be able to upload, view and download all files related to courses there are registered to.

        RAT: for a student to manage files efficiently.

        DEP: FR4

        \

        \textbf{3.1.16 Functional requirement 1.16}

        \textbf{ID: FR16}

        TITLE: Course/session Message system

        DESC: Given that a student is logged in, the student should be able to post/read questions and answers in courses/sessions he/she is registered to.

        RAT: for a student to post/read messages in course/session.

        DEP: FR4, FR8, FR10

        \

        \textbf{3.1.17 Functional requirement 1.17}

        \textbf{ID: FR17}

        TITLE: Course/session Rating system

        DESC: Given that a student is logged in, the student should be able to rate messages in courses/sessions he/she is registered to.

        RAT: for a student to rate messages in course/session.

        DEP: FR4, FR8, FR10, FR16

        \

        \textbf{3.1.18 Functional requirement 1.18}

        \textbf{ID: FR18}

        TITLE: Upload session video

        DESC: Given that a student is logged in, the student should be able to upload a video in sessions he/she is registered to.

        RAT: for a student to upload session videos.

        DEP: FR4, FR10

        \

        \textbf{3.1.19 Functional requirement 1.19}

        \textbf{ID: FR19}

        TITLE: Rate session

        DESC: Given that a student is logged in, the student should be able to rate a sessions he/she is registered to.

        RAT: for a student to rate session.

        DEP: FR4, FR10


        \newpage


        \textbf{3.2}\ \ \textbf{User Class 2 - The Teacher}

        \

        \textbf{3.2.1 Functional requirement 2.1}

        \textbf{ID: FR1}

        TITLE: Download the mobile application

        DESC: A teacher should be able to download the mobile application
        through either an application store or similar service on the mobile
        phone. The application should be free of charge.

        RAT: For a teacher to download the mobile application.

        DEP: None

        \

        \textbf{3.2.2 Functional requirement 2.2}

        \textbf{ID: FR2}

        TITLE: Download and notify teachers of new releases

        DESC: When a new/updated version or release of the software is released,
        the teacher should check for these manually. The download of the new
        release should be done through the mobile phone in the same way as
        downloading the mobile application.

        RAT: For a teacher to download the most updated release.

        DEP: FR1

        \

        \textbf{3.2.3 Functional requirement 2.3}

        \textbf{ID: FR3}

        TITLE: Teacher registration - Mobile and Web application

        DESC: Given that a teacher has downloaded the mobile application or has
        browsed to the platform's homepage, then the teacher should be able to
        register through the mobile application or the website. The teacher must
        provide first name, last name, password, e-mail address, university and
        the personal code he received from the university.

        RAT: For a teacher to register on the mobile application or the website.

        DEP: FR1

        \

        \textbf{3.2.4 Functional requirement 2.4}

        \textbf{ID: FR4}

        TITLE: Teacher log-in - Mobile and web application

        DESC: Given that a teacher has registered, then the teacher should be
        able to log into the application. The teacher's log-in information will
        be stored on the phone or browser, and in the future the teacher should
        be logged in automatically.

        RAT: For a teacher to be logged-in after he/she is registered.

        DEP: FR1, FR3

        \

        \
\textbf{3.2.5 Functional requirement 2.5}

        \textbf{ID: FR5}

        TITLE: Retrieve password

        DESC: Given that a teacher has registered, then the teacher should be
        able to retrieve his/her password by e-mail.

        RAT: For a teacher to retrieve his/her password.

        DEP: FR4

        \

        \textbf{3.2.6 Functional requirement 2.6}

        \textbf{ID: FR6}

        TITLE: Mobile/Web application - Homepage

        DESC: Given that a teacher is logged in to the application, then the
        first page that is shown should be the homepage, with the teacher's
        profile picture on the navigation bar.

        RAT: For a teacher to see his/her personal homepage.

        DEP: FR4


        \

        \textbf{3.2.7 Functional requirement 2.7}

        \textbf{ID: FR7}

        TITLE: Personal profile page

        DESC: Given that a teacher is logged in, then the teacher should be able
        to view his/her personal profile page.

        RAT: For a teacher to see his/her personal profile page.

        DEP: FR4

        \

        \textbf{3.2.8 Functional requirement 2.8}

        \textbf{ID: FR8}

        TITLE: Create Session

        DESC: Given that a teacher is logged in, the student should be able to create a session for other students to connect to.

        RAT: For a student to create sessions.

        DEP: FR4

        \

        \textbf{3.2.9 Functional requirement 2.9}

        \textbf{ID: FR9}

        TITLE: Connect to Session

        DESC: Given that a teacher is logged in, the student should be able to connect to a session.

        RAT: For a student to connect to sessions.

        DEP: FR4, FR9

        \

        \textbf{3.2.10 Functional requirement 2.10}

        \textbf{ID: FR10}

        TITLE: Personal course page

        DESC: Given that a teacher is logged in, then the teacher should be able
        to view his/her personal courses page, sorted by universities and
        departments.

        RAT: For a teacher to see his/her personal course page.

        DEP: FR4

        \

        \textbf{3.2.11 Functional requirement 2.11}

        \textbf{ID: FR11}

        TITLE: Free-text search

        DESC: Given that a teacher is logged in, then the teacher should be able
        to conduct a search by providing either course name, first name, last
        name or university name. The results are displayed in a list view by
        default.

        RAT: For a teacher to search through the free-text search.

        DEP: FR4

        \

        \textbf{3.2.12 Functional requirement 2.12}

        \textbf{ID: FR12}

        TITLE: Sorting results

        DESC: Given that a teacher has conducted a search, the teacher should be
        able to sort the results according to the result type: teacher, student, course or university.

        RAT: For a teacher to sort his/her search.

        DEP: FR11

        \

        \textbf{3.2.13 Functional requirement 2.13}

        \textbf{ID: FR13}

        TITLE: Filtering results

        DESC: Given that a teacher has conducted a search, the teacher should be
        able to filter the results according to the result type: teacher, student, course or university.

        RAT: For a teacher to filter his/her search.

        DEP: FR11

        \

        \textbf{3.2.14 Functional requirement 2.14}

        \textbf{ID: FR14}

        TITLE: Personal students list page

        DESC: Given that a teacher is logged in, then the teacher should be able
        to view his/her personal students list page, sorted by course.

        RAT: For a teacher to see his/her personal students list page.

        DEP: FR4

        \

        \textbf{3.2.15 Functional requirement 2.15}

        \textbf{ID: FR15}

        TITLE: Message all students in course

        DESC: Given that a teacher is logged in, then the teacher should be able
        to broadcast a message to all of his/her students in a selected course.

        RAT: For a teacher to broadcast a message to his/her students in a certain course.

        DEP: FR4

        \

        \textbf{3.2.16 Functional requirement 2.16}

        \textbf{ID: FR16}

        TITLE: Course/session Message system

        DESC: Given that a teacher is logged in, the teacher should be able to post/read questions and answers in courses/sessions he/she is registered to.

        RAT: for a teacher to post/read messages in course/session.

        DEP: FR4, FR8, FR10

        \

        \textbf{3.2.17 Functional requirement 2.17}

        \textbf{ID: FR17}

        TITLE: Course/session Rating system

        DESC: Given that a teacher is logged in, the teacher should be able to rate messages in courses/sessions he/she is registered to.

        RAT: for a teacher to rate messages in course/session.

        DEP: FR4, FR8, FR10, FR16

        \
\end{flushleft}

    \newpage
    \section{Failed approaches}
    \subsection{Real-Time Communication}
    wizeUp, serving as a messaging and live stream platform must be a real time application, all users need to be synchronized on all events. To satisfy this requirement, Initially the application would fetch the data from the server every 5 seconds. Although this approach worked it was inefficient in terms of overloading the server as well as the application itself. Not every 5 seconds something changes, and sometimes 5 seconded is to long. in addition fetching all the data every time is not necessary and time-consuming. Therefor we changed the approach and decided to work with a more efficient framework. we found the socket.io framework which made the Real-Time Communication much more efficient and easer to implement. Also, integrating the framework with the existing technologies we used in our project, was very convenient.

    \subsection{Authentication}
    When we first started implementing the project the only form of Authentication was upon login. After several courses in Cyber security we learned that this approach was incorrect as it made users' privacy vulnerable and allowed many other forms of attacks. In light of this we decided to verify users on every request to the server using jwt.


    \newpage
    \section{Future Work}

    \subsection{Vision}

    wizeUp will create a global academic community where teachers and students can enhance their teaching and studying skills through a social academic network.

    \subsection{Functionalities}
    \begin{itemize}
        \item Quiz:

        Quiz feature \cite{benefitsonlinelearning} in session containing two options a multiple choice (4 answers) or a true or false question (2 answers), the questions should be time bounded. at the end of the time each student will see if  his result was correct, and the teacher an analysis of the class's answers.


        \item Credit system:

        Many operations performed in wizeUp should reward the user with credit points. these points should be used in school related stores to purchase or discount items.

        \item Friends System:

        wizeUp is meant to be social as well, therefore, each user should not only be able to use wizeUp for educational purposes but also for social purposes \cite{socialnetwork}.

        \item Live Stream:

        Live Stream sessions to all users of that session. If possible save the stream as well for future viewing.

        \item Inbox Messages:

        wizeUp should allow users to send messages, share files, videos and links with each other.

        \item Submission Box + course grading:

        Each course holds the assignments' grading of each student and  the average for that course.
        courses should hold a Submission Box as well for exams and assignments \cite{flippedlearning, classroomofone}.

    \end{itemize}


    \newpage
    \section{Evaluation and results}
    \subsection{Evaluation:}
    \begin{itemize}
        \item QA: Each component that was built by a team member was tested by {\textbf{ALL}} other team members. The tests consisted of code review, efficiency enhancement and integration.

        \item Feedback: The platform was delivered to several individuals for review and feedback about the user experience.

        \item Simulation: A demo class was dedicated to use the platform in several scenarios using various devices and environments.
    \end{itemize}


    \subsection{Results:}
    The evaluation resulted in many conclusions about the functionality, code efficiency, unresolved bugs and user experience. We tended to all the problems and improved.



    \newpage
    \section{Conclusion}

    Although wizeUp is far from perfect, we brought it from an idea to a working cross platform application. The project consists of many different frame works and technologist which we had to learn on our own. We divided the work among ourselves and each team member was responsible for a different component. The working process was logged and tracked using git. We enjoined working together and learned allot thanks to our supervisor {\underline{Dr. Amit Dvir}}.

    \

    \

    \

    {\centering
    {\huge \textbf{Start learning wizer.}}

    }


    \newpage
    \bibliographystyle{unsrt}
    \bibliography{references}

\end{document}
